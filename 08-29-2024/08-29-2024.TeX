\documentclass[12pt]{article}
\usepackage{geometry}
\geometry{a4paper, margin=1in}
\usepackage{setspace}
\onehalfspacing
\usepackage{titlesec}
\titleformat{\section}{\large\bfseries}{\thesection}{1em}{}

\begin{document}

\title{Environmental Hazards and Policy Considerations}
\author{}
\date{August 29, 2024}
\maketitle

\section{Soil Creep}
Soil creep refers to the gradual downhill movement of soil and rock, often occurring on slopes. This process can be slow, ranging from inches per decade to several feet per year, depending on the terrain and underlying conditions.

Key indicators of soil creep include:
\begin{itemize}
    \item Trees with curved trunks: The base tilts downhill, while the upper trunk grows vertically to correct itself.
    \item Cracks in walls or foundations: These may result from uneven settling of the structure.
\end{itemize}

The movement of soil poses risks to buildings situated on slopes. For instance, the front part of a house may remain stationary while the back shifts downhill, damaging the foundation and, in severe cases, leading to structural failure. To mitigate this, builders use measures such as buttressing or pile-driven supports to stabilize the back wall and prevent collapse. Note that steep hills cannot be effectively retained with low-foundation walls alone; advanced shoring techniques are necessary.

Landslides, a rapid form of soil creep, present a more immediate danger. Monitoring signs of creep can help predict and mitigate these risks.

\section{Flooding}
Flooding is influenced by two main factors:
\begin{enumerate}
    \item \textbf{Elevation:} Homes on higher ground are less likely to flood than those at lower elevations.
    \item \textbf{Proximity to coastlines or water bodies:} Properties closer to shorelines face higher flood risks, particularly during storms or heavy rainfall.
\end{enumerate}

Flood hazard maps, which consider these variables, are critical for assessing risk and planning construction or land use.

\section{Radon}
Radon is a radioactive gas produced by the decay of naturally occurring elements in soil and rock. While harmless in the open atmosphere, radon becomes dangerous when it accumulates indoors. It can seep into homes through cracks in foundations, groundwater, or drainage systems.

Health risks depend on exposure levels:
\begin{itemize}
    \item \textbf{High doses:} Can be fatal.
    \item \textbf{Moderate doses:} Increase the risk of lung cancer.
    \item \textbf{Low doses:} Effects are debated; some studies suggest harm, while others do not.
\end{itemize}

To mitigate radon exposure:
\begin{itemize}
    \item Ventilate basements with fresh air to prevent radon buildup.
    \item Treat water sources, such as wells, to remove radon by agitation or other methods.
\end{itemize}

Radon levels vary with soil composition; regions with high levels of uranium or similar elements in the ground are more susceptible.

\section{Wetlands Protection and Policy}
Wetlands provide critical environmental benefits, including:
\begin{itemize}
    \item Storing floodwater and reducing flood risks.
    \item Serving as habitats for fish and wildlife.
    \item Maintaining surface water flow during dry periods.
\end{itemize}

Under laws such as the \textit{Wetlands Protection Act}, using or developing wetlands is prohibited, even for property owners. Balancing private property rights with environmental conservation remains a contentious issue.

\section{Environmental Trade-offs and Recycling}
Environmental policies often involve trade-offs. For example:
\begin{itemize}
    \item Recycling is generally beneficial but can lead to localized pollution if improperly managed. Effective programs ensure waste is processed without harming the surrounding area.
    \item Decisions such as spraying pesticides or draining swamps require weighing public health against environmental impacts.
\end{itemize}

Local action, such as promoting efficient recycling practices and safeguarding wetlands, can reduce environmental harm without compromising community needs.

\end{document}
