\documentclass{article}

\usepackage{amsmath}
\usepackage{amssymb}
\usepackage{graphicx}
\usepackage{booktabs}

\begin{document}

\title{Continental Drift: Evidence and Sociological Implications}
\author{}
\date{}

\maketitle

\section*{Introduction}

Continental drift is the theory that the Earth's continents have moved over geological time. Proposed by Alfred Wegener in 1912, the theory suggests that the continents were once joined together in a single large landmass, which he called \emph{Pangaea}. Wegener's hypothesis was based on multiple lines of evidence, such as the fit of the continents, the alignment of mountain ranges, fossil distributions, and paleoclimatic data. Despite being initially ridiculed by the scientific community, his ideas laid the foundation for the modern theory of plate tectonics.

\section{Evidence for Continental Drift}

Wegener supported his theory with six key pieces of evidence:

\subsection{The Fit of the Continents}

One of the most striking pieces of evidence for continental drift is the apparent jigsaw-like fit of the continents. For example:
\begin{itemize}
    \item The east coast of South America closely aligns with the west coast of Africa.
    \item When continents are reassembled, their boundaries exhibit minimal gaps or overlaps.
\end{itemize}

\subsection{Alignment of Mountain Ranges}

Mountain ranges on different continents align when the continents are repositioned:
\begin{itemize}
    \item The Appalachian Mountains in North America connect seamlessly with the Atlas Mountains in Africa.
    \item Similar alignments are seen in mountain ranges across Greenland, Scotland, and Scandinavia.
\end{itemize}
These alignments suggest that the ranges were formed as part of the same geological structures before the continents drifted apart.

\subsection{Climate Anomalies}

Geological evidence indicates that continents have moved through different climatic zones:
\begin{itemize}
    \item \textbf{Coal deposits:} Found in Arctic regions such as Svalbard, despite coal forming in warm, swampy environments.
    \item \textbf{Glacial deposits and striations:} Evidence of ancient glaciers in tropical regions like Africa and South America.
    \item \textbf{Salt deposits:} Evaporites such as halite in regions like Michigan suggest a hot, arid climate in the past.
\end{itemize}

\subsection{Glacial Striations}

Striations—scratches left on rocks by glaciers—often indicate ice movement from ocean basins onto land, which is impossible. When continents are repositioned into a supercontinent, the striations make logical sense, pointing outward from a central ice sheet.

\subsection{Fossil Evidence}

Fossils of identical species are found on continents now separated by oceans:
\begin{itemize}
    \item \textbf{Mesosaurus:} A freshwater reptile found in both South America and Africa.
    \item \textbf{Glossopteris:} A plant species distributed across South America, Africa, India, and Antarctica.
\end{itemize}
The presence of these fossils implies that these landmasses were once connected.

\subsection{Geological Structures}

Rock formations of similar age and composition appear on continents separated by vast oceans. For instance:
\begin{itemize}
    \item Sedimentary layers in South America and Africa match in age and composition.
    \item Volcanic and metamorphic rocks of similar ages are found in North America and Europe.
\end{itemize}

\begin{table}[h!]
    \centering
    \caption{Summary of Evidence for Continental Drift}
    \begin{tabular}{@{}ll@{}}
        \toprule
        \textbf{Evidence Type}       & \textbf{Details} \\ 
        \midrule
        Fit of Continents            & Continents align like puzzle pieces, e.g., South America and Africa. \\ 
        Mountain Ranges              & Separate ranges align when continents are joined. \\ 
        Climate Anomalies            & Coal in Arctic regions, glacial evidence in tropics. \\ 
        Glacial Striations           & Logical flow directions when continents are reassembled. \\ 
        Fossil Evidence              & Identical fossils found on now-separated continents. \\ 
        Geological Structures        & Matching rock formations across continents. \\ 
        \bottomrule
    \end{tabular}
\end{table}

\section{The Initial Rejection of Wegener's Theory}

Despite its strong evidence, Wegener's theory faced widespread rejection. Critics cited:
\begin{itemize}
    \item \textbf{Lack of a mechanism:} Wegener could not explain how continents moved.
    \item \textbf{Resistance from authority:} Established scientists dismissed Wegener as a crackpot, and his supporters faced ostracism.
    \item \textbf{Sociological factors:} Fear of challenging authority figures and conformist attitudes hindered acceptance of new ideas.
\end{itemize}

\section{Sociological Lessons from Wegener's Story}

Wegener's experiences offer important lessons for scientific and societal progress:
\begin{itemize}
    \item \textbf{Evidence-based reasoning:} Ideas should be evaluated based on evidence, not reputation.
    \item \textbf{Open-mindedness:} Dismissing unconventional ideas delays progress.
    \item \textbf{Critical thinking:} Questioning authority fosters a culture of innovation and discovery.
\end{itemize}

\end{document}
