\documentclass[12pt]{article}
\usepackage{amsmath}
\usepackage{graphicx}
\usepackage{hyperref}

\title{Geological Time and Dating Methods}
\author{}
\date{}

\begin{document}
\maketitle

\section*{Introduction}
Understanding when geological events occurred is key to grasping Earth's processes and history. Geologists use two main methods to determine timing:

\section*{Relative and Absolute Time}
\subsection*{1. Relative Time}
\textbf{Definition:} Establishes the order of events without specifying exact ages. It reveals the sequence in which events occurred.

\begin{itemize}
    \item \textbf{Example:} Knowing that you are older than your sibling provides a relative comparison but no specific age.
    \item \textbf{Application:} By examining rock layers and their relationships, geologists determine the sequence of events when absolute dates are unavailable.
\end{itemize}

\subsection*{2. Absolute Time}
\textbf{Definition:} Assigns a numerical age to an event, providing a precise measurement.

\begin{itemize}
    \item \textbf{Example:} Stating "I am 22 years old" gives an exact age.
    \item \textbf{Application:} Techniques like radiometric dating allow scientists to calculate the age of rocks, fossils, or events.
\end{itemize}

\section*{Principles of Relative Time}
Geologists use several principles to determine the relative ages of rock layers and features:

\subsection*{Principle of Superposition}
\begin{itemize}
    \item \textbf{Definition:} In undisturbed sedimentary rock layers, the oldest layers lie at the bottom, and the youngest are on top.
    \item \textbf{Explanation:} Newer layers form on top of older ones as sediment accumulates.
    \item \textbf{Example:} Snowfall creates distinct layers, with older snow at the bottom if undisturbed.
    \item \textbf{Limitation:} This principle applies only to undisturbed layers. Processes like mountain building can deform and overturn layers.
\end{itemize}

\subsection*{Principle of Original Horizontality}
\begin{itemize}
    \item \textbf{Definition:} Sedimentary layers are originally deposited horizontally.
    \item \textbf{Application:} Tilted or folded layers indicate disturbances after deposition.
\end{itemize}

\subsection*{Principle of Cross-Cutting Relationships}
\begin{itemize}
    \item \textbf{Definition:} Features cutting across rocks are younger than the rocks they disrupt.
    \item \textbf{Examples:}
    \begin{itemize}
        \item \textbf{Joints:} Fractures in rocks without significant movement.
        \item \textbf{Faults:} Fractures where rocks move, often causing earthquakes.
        \item \textbf{Igneous Dikes:} Magma intrusions solidified after cutting through rock layers.
    \end{itemize}
    \item \textbf{Application:} Dating cross-cutting features helps reconstruct geological sequences.
\end{itemize}

\subsection*{Principle of Faunal and Floral Succession}
\begin{itemize}
    \item \textbf{Definition:} Fossils occur in a predictable order.
    \item \textbf{Application:} Fossil assemblages help date and correlate rock layers globally.
    \item \textbf{Significance:} This principle reveals Earth's biological and geological history.
\end{itemize}

\section*{The Geologic Column}
\begin{itemize}
    \item \textbf{Definition:} A diagram combining global rock layer data to represent Earth's geological history.
    \item \textbf{Representation:} A vertical sequence of rock layers, with the oldest at the bottom.
    \item \textbf{Importance:}
    \begin{itemize}
        \item Acts as a reference for major geological and biological events.
        \item Helps correlate rock layers worldwide.
    \end{itemize}
\end{itemize}

\section*{Absolute Dating Methods}
\subsection*{Annual Layers in Ice or Sediment}
\begin{itemize}
    \item \textbf{Ice Cores:} Counting annual layers in glaciers and ice sheets determines age.
    \item \textbf{Sediment Layers (Varves):} Sediment layers in lakes also provide annual records.
\end{itemize}

\subsection*{Radiometric Dating}
\begin{itemize}
    \item \textbf{Definition:} Measures radioactive decay in minerals to determine age.
    \item \textbf{Key Concepts:}
    \begin{itemize}
        \item \textbf{Radioactivity:} Unstable elements decay into stable ones over time.
        \item \textbf{Half-Life:} The time for half of a radioactive element to decay.
        \item \textbf{Parent Isotope:} The original radioactive element.
        \item \textbf{Daughter Isotope:} The decay product.
    \end{itemize}
\end{itemize}

\subsection*{Common Isotopes Used}
\begin{itemize}
    \item \textbf{Uranium-Lead Dating:}
    \begin{itemize}
        \item Uranium-238 $\rightarrow$ Lead-206.
        \item Half-life: 4.5 billion years.
        \item Application: Dating Earth's oldest rocks.
    \end{itemize}
    \item \textbf{Potassium-Argon Dating:}
    \begin{itemize}
        \item Potassium-40 $\rightarrow$ Argon-40.
        \item Half-life: 1.25 billion years.
        \item Application: Dating igneous and metamorphic rocks.
    \end{itemize}
    \item \textbf{Carbon-14 Dating:}
    \begin{itemize}
        \item Carbon-14 $\rightarrow$ Nitrogen-14.
        \item Half-life: 5,730 years.
        \item Application: Dating recent organic remains (up to 50,000 years).
    \end{itemize}
\end{itemize}

\subsection*{Limitations of Radiometric Dating}
\begin{itemize}
    \item Accuracy decreases after many half-lives as the remaining parent isotope becomes too small to measure.
    \item Requires a closed system with no gain or loss of isotopes.
    \item Not all rocks can be dated directly (e.g., sedimentary rocks are composites).
\end{itemize}

\section*{The Age of the Earth}
\begin{itemize}
    \item \textbf{Current Estimate:} About 4.5 billion years.
    \item \textbf{Evidence Sources:}
    \begin{itemize}
        \item \textbf{Earth's Oldest Rocks:} Radiometric dating of zircon shows ages up to 4.4 billion years.
        \item \textbf{Meteorites:} Dating consistently gives ages around 4.5 billion years.
        \item \textbf{Lunar Samples:} Moon rocks also date to about 4.5 billion years.
    \end{itemize}
\end{itemize}

\end{document}
