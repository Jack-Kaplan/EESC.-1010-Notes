\documentclass[12pt]{article}
\usepackage{amsmath}
\usepackage{graphicx}
\usepackage{hyperref}

\title{Earth's Spheres and Rock Processes}
\author{}
\date{}

\begin{document}
\maketitle

\section*{Earth's Spheres}
\begin{itemize}
    \item \textbf{Atmosphere}: The gases surrounding Earth.
    \item \textbf{Hydrosphere}: All surface water.
    \item \textbf{Biosphere}: Living organisms and their support systems.
    \item \textbf{Solid Earth}: Earth's mostly solid rock.
    \item \textbf{Regolith}: Loose material covering solid rock (e.g., soil, sand, boulders).
\end{itemize}
\textit{Note: These zones are discontinuous and lack sharp boundaries.}

\section*{Rocks and Minerals}
\subsection*{What is a Rock?}
\begin{itemize}
    \item A collection of minerals attached to each other.
\end{itemize}

\subsection*{What is a Mineral?}
\begin{itemize}
    \item A mineral is:
    \begin{itemize}
        \item Naturally occurring.
        \item Inorganic.
        \item Crystalline.
    \end{itemize}
    \item Substances consist of atoms, arranged either:
    \begin{itemize}
        \item Randomly, or
        \item Uniformly, in a repeating pattern.
    \end{itemize}
\end{itemize}

\subsection*{What is a Crystal?}
\begin{itemize}
    \item A solid with a geometric shape and flat surfaces, formed by its atomic structure.
    \item Growth of crystals is determined by their atomic-scale arrangement.
\end{itemize}

\subsection*{Examples of Substances}
\begin{itemize}
    \item \textbf{Coal}:
    \begin{itemize}
        \item Naturally occurring: Yes.
        \item Inorganic: No.
        \item Crystalline: Not relevant.
    \end{itemize}
    \item \textbf{Ice}:
    \begin{itemize}
        \item Naturally occurring: Yes.
        \item Inorganic: Yes.
        \item Crystalline: Yes (e.g., snowflakes).
    \end{itemize}
\end{itemize}

\section*{Processes Affecting Rock}
\subsection*{External Processes}
\subsubsection*{Weathering}
\begin{itemize}
    \item \textbf{Physical Weathering}: 
    \begin{itemize}
        \item Rock breaks into smaller pieces without chemical change.
    \end{itemize}
    \item \textbf{Chemical Weathering}:
    \begin{itemize}
        \item Rock undergoes chemical changes, forming new minerals or dissolving substances (e.g., rusting iron, dissolving salt).
        \item Soil forms largely through chemical weathering.
    \end{itemize}
\end{itemize}

\subsubsection*{Erosion}
\begin{itemize}
    \item The movement of earth material as particles or in solution, via:
    \begin{itemize}
        \item Rivers.
        \item Glaciers.
    \end{itemize}
\end{itemize}

\subsubsection*{Deposition}
\begin{itemize}
    \item The accumulation of material transported by erosion.
\end{itemize}

\subsection*{Internal Processes}
\begin{itemize}
    \item Earthquakes.
    \item Volcanoes.
    \item Mountain building.
\end{itemize}

\end{document}
