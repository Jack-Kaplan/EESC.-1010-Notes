\documentclass[12pt]{article}
\usepackage{amsmath}
\usepackage{graphicx}
\usepackage{hyperref}

\title{Earth Structure: Interior and Origin}
\author{}
\date{October 31, 2024}

\begin{document}
\maketitle

\section*{Introduction: Questions to Be Answered}
\begin{itemize}
    \item Earth is a roughly spherical planet with a radius of about 4,000 miles. Composed largely of rock, it orbits the Sun.
    \item This paper addresses three key questions:
    \begin{itemize}
        \item How do scientists know what lies deep inside Earth, given the impossibility of direct sampling?
        \item Does Earth consist of distinct zones or layers? If so, what are they made of?
        \item Is Earth entirely solid, or do its layers include other states of matter, such as liquid or gas?
    \end{itemize}
    \item To illustrate the concept of layering, consider a baseball: it has distinct layers of leather, cork, stitching, and rubber.
\end{itemize}

---

\section*{Why Is Geology Less Popular in Urban Areas Like New York?}
\begin{itemize}
    \item New York City lacks dramatic geological features, such as mountains or canyons, that might spark interest in Earth science.
    \item The urban environment, dominated by concrete structures and cultural activities like sports and entertainment, overshadows natural history.
    \item As one geology professor quipped in the \textit{New York Times}, \textit{"I'd rather attend a Yankees game than teach geology!"}
\end{itemize}

---

\section*{Earth's Layers}
\begin{itemize}
    \item \textbf{Crust:} The outermost layer, thin and fragile, is 7–30 miles thick and made of lighter silicate rocks.
    \item \textbf{Mantle:} This layer stretches from about 20 miles beneath the surface to a depth of 1,800 miles. It is denser and consists of heavier silicates.
    \item \textbf{Core:} The Earth's core lies below the mantle, extending to the planet's center, about 4,000 miles deep. It is denser than the mantle and composed mostly of metal.
    \item A useful mnemonic for Earth's layers: \textit{Thin, Thick, Thicker}.
\end{itemize}

---

\section*{How Do Scientists Identify Earth's Layers?}
\begin{itemize}
    \item Earthquake waves spread through the planet, changing speed and direction when they hit different materials.
    \item By analyzing how these waves echo or refract, scientists calculate the boundaries and thicknesses of Earth's layers.
    \item This is similar to measuring the distance to a wall by timing the return of a bouncing ball.
\end{itemize}

---

\section*{How Do Scientists Determine the Composition of Each Layer?}
\subsection*{Direct Sampling Attempts}
\begin{itemize}
    \item The deepest human-made hole, the \textbf{Kola Superdeep Borehole} in Russia, reaches 12,262 meters—barely penetrating the crust.
    \item Earth's continental crust is thicker and less dense, resembling granite. By contrast, the oceanic crust is thinner and denser, resembling basalt.
    \item Current drilling focuses on oceanic crust, as its reduced thickness offers a better chance of reaching the mantle.
\end{itemize}

\subsection*{Mantle Composition}
\begin{itemize}
    \item \textbf{Mantle Xenoliths:} Solid fragments of mantle rock, carried to the surface by volcanic activity, provide direct evidence of its composition.
    \item The mantle consists largely of minerals such as \textbf{olivine} and \textbf{pyroxene}, with traces of others.
    \item \textbf{Ophiolite Complexes:} Sections of oceanic crust, pushed onto land during tectonic collisions, indirectly reveal the mantle's makeup.
\end{itemize}

\subsection*{Confirming Mantle Composition}
\begin{itemize}
    \item Laboratory tests show that sound waves travel through mantle minerals like olivine and pyroxene at speeds matching seismic data from earthquakes.
    \item \textbf{Meteorites:} These remnants of asteroids, the solar system's building blocks, support theories about Earth's internal composition.
\end{itemize}

---

\section*{The Core's Composition}
\begin{itemize}
    \item Earth's total mass, combined with the masses of the crust and mantle, allows scientists to deduce the core's mass.
    \item The core's density can be estimated from its mass and volume:
    \[
    V = \frac{4}{3} \pi r^3 \quad \text{where } r \approx 3,486 \text{ miles}.
    \]
    \item This density, roughly 13.1 grams per cubic centimeter, closely matches that of iron.
    \item Evidence suggests the core is largely iron, supported by:
    \begin{itemize}
        \item Its high density.
        \item The presence of Earth's magnetic field, which is generated by movements of molten iron in the outer core.
    \end{itemize}
\end{itemize}

---

\section*{Why Does Earth Experience Seasons?}
\begin{itemize}
    \item Earth’s axis is tilted at an angle of 23.5° relative to its orbit around the Sun, leading to seasonal variations in sunlight.
    \item The tilt originated from a collision with a Mars-sized object during Earth's early formation.
    \item As Earth solidified, this tilt became fixed, creating the seasonal cycle.
\end{itemize}

\end{document}
