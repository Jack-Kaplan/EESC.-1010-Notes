\documentclass[12pt]{article}
\usepackage{amsmath}
\usepackage{graphicx}
\usepackage{hyperref}

\title{Policy, Media, and Science in Geology}
\author{}
\date{}

\begin{document}
\maketitle

\section*{Introduction}
\textit{This lecture extends beyond geology to explore the intersection of policy, media, and science, reflecting the professor's broader interests.}

\section*{Policy and Climate Change}
\subsection*{Policy Formation}
\begin{itemize}
    \item Climate policies, according to the professor, are often created without properly informing the public about:
    \begin{itemize}
        \item The underlying evidence.
        \item The scientific basis.
        \item The associated trade-offs.
    \end{itemize}
    \item Instead, the public is frequently directed on "what to think" rather than encouraged to engage critically.
\end{itemize}

\subsection*{Who Sets Policy?}
\begin{itemize}
    \item \textbf{Key Stakeholders}:
    \begin{itemize}
        \item Governments.
        \item Industry representatives.
        \item Environmental organizations.
    \end{itemize}
    \item \textbf{Observations}:
    \begin{itemize}
        \item Each group has vested interests, which often diverge from those of the general public.
        \item Public involvement in policymaking is minimal.
    \end{itemize}
\end{itemize}

\section*{Media and Public Accountability}
\subsection*{Discussion on News Sources}
The professor encourages students to evaluate news critically, contrasting two statements:
\begin{itemize}
    \item \textit{"Millions of people are slowly poisoning themselves sip by sip, with the authorities confounded about what to do."}
    \item \textit{"These wells and pumps not only have helped the country to become self-sufficient in rice production, but have saved millions of people from disease and death."}
\end{itemize}
Students are asked to identify which statement originates from the \textbf{New York Post} (NYP) and which from the \textbf{New York Times} (NYT).

\subsection*{Public Role and Media Responsibility}
\begin{itemize}
    \item \textbf{The Professor’s View}:
    \begin{itemize}
        \item Only the public can hold policymakers to account.
        \item The media, meant to serve as a watchdog, often falls short of this responsibility.
    \end{itemize}
    \item \textbf{Counterpoint}:
    \begin{itemize}
        \item Investigative groups such as ProPublica demonstrate that some journalism remains rigorous.
        \item Suggesting everyone should analyze raw data or act as journalists is neither practical nor effective.
    \end{itemize}
\end{itemize}

\section*{What is Science?}
\subsection*{Definition and Misconceptions}
\begin{itemize}
    \item Science is often presented as a rigid process of defined steps.
    \item The professor critiques this oversimplification, emphasizing that science is a dynamic and iterative process.
\end{itemize}

\subsection*{Criticism of NYPIRG}
\begin{itemize}
    \item The professor alleges NYPIRG has engaged in:
    \begin{itemize}
        \item Research misconduct.
        \item Misappropriation of funding.
        \item Falsely reporting higher illness rates in Starrett City linked to nearby landfills.
    \end{itemize}
\end{itemize}

\section*{Elements of the Scientific Method}
\subsection*{Avoiding Bias in Data Collection}
\begin{itemize}
    \item \textbf{Physical Measurements}:
    \begin{itemize}
        \item Use modern and properly calibrated equipment to ensure accuracy.
    \end{itemize}
    \item \textbf{Surveys}:
    \begin{itemize}
        \item The professor provides an example: \textit{"How long have you lived in Starrett City, and have you experienced any of these illnesses?"}
        \item Critique: This is not a valid double-blind survey because:
        \begin{itemize}
            \item The surveyor knows the purpose and desired outcome.
            \item A true double-blind method requires that neither the surveyor nor the participant knows the study's purpose.
        \end{itemize}
    \end{itemize}
\end{itemize}

\subsection*{Statistical Significance}
\begin{itemize}
    \item Data must meet statistical significance criteria to draw valid scientific conclusions.
\end{itemize}

\end{document}
