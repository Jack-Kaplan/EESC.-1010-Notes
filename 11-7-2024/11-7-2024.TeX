\documentclass{article}
\usepackage{amsmath}
\usepackage{graphicx}
\usepackage{array}
\usepackage{tikz}
\usetikzlibrary{shapes, arrows, positioning, matrix}

\title{Earth's Structure and Dynamics}
\date{}

\begin{document}
\maketitle

\section*{Meteorite Insights: Understanding Earth's Composition}

Meteorites and planets share similar materials, offering a window into Earth's makeup. Although direct study of Earth's interior is impossible, meteorites provide indirect evidence, having been naturally fragmented and made accessible by impacts in space and on Earth.

\subsection*{Key Points}
\begin{itemize}
    \item \textbf{Meteorites as Samples of Earth:} When planets are shattered, their debris includes fragments of the crust, mantle, and core. These meteorites allow scientists to analyze the building blocks of terrestrial planets, including Earth.
    \item \textbf{Relevance to Earth's Composition:} The similarities between Earth and meteorites suggest that studying these fragments helps us deduce the mineral and chemical makeup of Earth's internal layers.
\end{itemize}

\section*{Seismic Analysis: Decoding Earth's Internal Structure}

Seismic waves generated by earthquakes reflect and refract at boundaries between Earth's layers. These waves, akin to "echoes," enable scientists to map the structure, composition, and physical properties of these layers.

\subsection*{Highlights}
\begin{itemize}
    \item \textbf{Reflection and Refraction:} Seismic waves change direction and speed at the boundaries between different geological layers, such as the crust and mantle. By analyzing these behaviors, researchers identify distinct internal zones.
    \item \textbf{Example: The Mohorovičić Discontinuity (Moho):} This boundary, marking the division between the crust and mantle, is identified through seismic wave reflections.
\end{itemize}

\section*{Earth's Layers: Physical and Chemical Perspectives}

Earth's internal structure is organized into layers distinguished by physical properties and chemical composition. These layers reveal the dynamic and complex nature of the planet.

\subsection*{Chemical Layers}
\begin{enumerate}
    \item \textbf{Crust:} The outermost layer, composed primarily of silicate minerals.
    \item \textbf{Mantle:} The thick, intermediate layer, consisting of silicate materials enriched in iron and magnesium.
    \item \textbf{Core:} Divided into the liquid outer core and solid inner core, both dominated by iron with traces of nickel.
\end{enumerate}

\subsection*{Physical Layers}
\begin{enumerate}
    \item \textbf{Lithosphere:} 
        \begin{itemize}
            \item \textbf{Composition:} Comprises the crust and the rigid upper mantle.
            \item \textbf{Properties:} Acts as Earth's solid, brittle outer shell.
        \end{itemize}
    \item \textbf{Asthenosphere:} 
        \begin{itemize}
            \item \textbf{Composition:} Located beneath the lithosphere in the upper mantle.
            \item \textbf{Properties:} Partially molten and capable of slow flow, similar to viscous dough.
        \end{itemize}
    \item \textbf{Mesosphere (Lower Mantle):} 
        \begin{itemize}
            \item \textbf{Composition:} Extends to the outer core.
            \item \textbf{Properties:} Solid due to immense pressure.
        \end{itemize}
    \item \textbf{Outer Core:}
        \begin{itemize}
            \item \textbf{Composition:} Molten iron with small amounts of nickel.
            \item \textbf{Properties:} A fluid layer that generates Earth's magnetic field through convection.
        \end{itemize}
    \item \textbf{Inner Core:}
        \begin{itemize}
            \item \textbf{Composition:} Solid iron and nickel.
            \item \textbf{Properties:} Extremely dense and under immense pressure.
        \end{itemize}
\end{enumerate}

\subsection*{Layer Summary Table}

\begin{table}[h!]
\centering
\begin{tabular}{|l|l|l|}
\hline
\textbf{Layer} & \textbf{Physical State} & \textbf{Chemical Composition} \\
\hline
Lithosphere & Solid & Crust + Upper Mantle \\
Asthenosphere & Plastic & Upper Mantle \\
Mesosphere & Solid & Lower Mantle \\
Outer Core & Liquid & Molten Iron \\
Inner Core & Solid & Solid Iron \\
\hline
\end{tabular}
\caption{Summary of Earth's Physical and Chemical Layers}
\end{table}

\section*{Seismic Waves: Probing Earth's Interior}

Seismic waves, generated by earthquakes, are vital tools for studying Earth's interior. They reveal differences in material properties, such as density and elasticity.

\subsection*{Types of Seismic Waves}
\begin{itemize}
    \item \textbf{P-Waves (Primary Waves):}
        \begin{itemize}
            \item \textbf{Characteristics:} Fast-moving waves that travel through solids, liquids, and gases.
            \item \textbf{Role:} Provide data on the compressional properties of Earth's materials.
        \end{itemize}
    \item \textbf{S-Waves (Secondary Waves):}
        \begin{itemize}
            \item \textbf{Characteristics:} Slower waves that propagate only through solids.
            \item \textbf{Role:} Their inability to pass through liquids confirms the existence of the molten outer core.
        \end{itemize}
\end{itemize}

\section*{Faults: Evidence of Earth's Tectonic Forces}

Faults form when rocks fracture under stress. These structures reveal the nature and intensity of tectonic forces shaping Earth's crust.

\subsection*{Types of Faults}
\begin{enumerate}
    \item \textbf{Normal Faults:}
        \begin{itemize}
            \item \textbf{Movement:} The hanging wall slides downward relative to the footwall.
            \item \textbf{Cause:} Tensional forces that stretch Earth's crust.
            \item \textbf{Example:} The Great Rift Valley in East Africa.
        \end{itemize}
    \item \textbf{Reverse Faults:}
        \begin{itemize}
            \item \textbf{Movement:} The hanging wall moves upward relative to the footwall.
            \item \textbf{Cause:} Compressional forces from tectonic collisions.
            \item \textbf{Example:} The formation of the Himalayas.
        \end{itemize}
    \item \textbf{Strike-Slip Faults:}
        \begin{itemize}
            \item \textbf{Movement:} Rocks slide horizontally past each other.
            \item \textbf{Example:} The San Andreas Fault in California.
        \end{itemize}
\end{enumerate}

\section*{Summary}

Earth's structure is a dynamic interplay of physical states and chemical compositions. Seismic waves and fault analysis reveal the processes shaping the planet. Understanding these features offers valuable insights into Earth's past and present geological activity.

\end{document}
