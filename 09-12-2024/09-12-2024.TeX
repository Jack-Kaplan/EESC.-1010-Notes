\documentclass[12pt]{article}
\usepackage{amsmath}
\usepackage{graphicx}
\usepackage{hyperref}
\usepackage{listings}

\title{Geological Cycles and Processes}
\author{}
\date{}

\begin{document}
\maketitle

\section*{Geological Cycles}
\subsection*{Definition of Cycles}
\begin{itemize}
    \item \textbf{Cycles}: Recurring events, such as the seasons.
    \item \textbf{Geological Cycles}: Recurring processes involving Earth materials.
\end{itemize}

\section*{Examples of Geological Cycles}
\subsection*{The Water Cycle}
\begin{itemize}
    \item A sequence of events involving the Earth's water.
    \item \textbf{Key Stages of the Water Cycle}:
    \begin{itemize}
        \item Ocean evaporation $\rightarrow$ vapor transport $\rightarrow$ precipitation $\rightarrow$ runoff $\rightarrow$ infiltration $\rightarrow$ runoff again $\rightarrow$ ocean $\rightarrow$ evaporation.
        \item Transpiration can also occur during the cycle.
    \end{itemize}
    \item \textbf{Transpiration}:
    \begin{itemize}
        \item Plants absorb water through roots, transport it, and release it as vapor.
        \item Often considered alongside evaporation as part of the cycle.
    \end{itemize}
\end{itemize}

\subsection*{Aspects of the Water Cycle}
\textbf{Steady State (Dynamic Equilibrium)}:
\begin{itemize}
    \item When input equals output, the system remains stable.
    \item Examples:
    \begin{itemize}
        \item A pool with balanced inflow and evaporation stays level.
        \item An oven in equilibrium loses as much heat as it gains.
    \end{itemize}
    \item Feedback mechanisms help natural systems maintain equilibrium.
    \item Example: A bucket with water inflow and an outflow hole stabilizes when inflow matches outflow.
\end{itemize}

\textbf{Oceans in Steady State}:
\begin{itemize}
    \item Oceans achieve balance when water inflow equals outflow.
    \item Factors include:
    \begin{itemize}
        \item Runoff, infiltration, and precipitation balance evaporation.
        \item Land effects are accounted for through runoff, infiltration, and precipitation.
    \end{itemize}
\end{itemize}

\section*{The Rock Cycle}
\begin{itemize}
    \item Begins with \textbf{uplift}, exposing igneous rock at the surface as an outcrop.
    \item Key processes:
    \begin{itemize}
        \item Weathering, erosion, transport, and deposition $\rightarrow$ sediment.
        \item Sediment may:
        \begin{itemize}
            \item Be uplifted to form an outcrop.
            \item Be buried, leading to compaction and cementation $\rightarrow$ sedimentary rock.
        \end{itemize}
        \item Sedimentary rock may:
        \begin{itemize}
            \item Be uplifted.
            \item Be buried further $\rightarrow$ heat and pressure $\rightarrow$ metamorphic rock.
        \end{itemize}
        \item Metamorphic rock may:
        \begin{itemize}
            \item Be uplifted.
            \item Melt $\rightarrow$ magma $\rightarrow$ igneous rock (intrusive or extrusive).
        \end{itemize}
    \end{itemize}
    \item The cycle operates continuously and simultaneously worldwide.
\end{itemize}

\section*{Chemical Cycles}
\begin{itemize}
    \item Include cycles for nitrogen, sulfur, carbon, phosphorus, and salts.
\end{itemize}

\subsection*{The Salt Cycle}
\begin{itemize}
    \item Weathering dissolves sodium chloride (NaCl), transporting it to the ocean via rivers or groundwater.
    \item Saltwater is buried with sediments.
    \item Uplift brings these sediments and rocks back to land, completing the cycle.
\end{itemize}

\subsection*{The Carbon Cycle}
\begin{itemize}
    \item No further study of the Carbon Cycle is required at this stage.
\end{itemize}

\section*{Global Warming}
\textit{Nearly all aspects of life depend on processes that emit molecules with uneven electrical charges, which absorb and re-emit thermal radiation. This increases the frequency and intensity of extreme weather events, leading to more loss of life, property, droughts, and famine. Harmful practices often become more common as a result. Humanity currently releases about 51 billion tons of greenhouse gases annually. The aim must be to reduce this to zero.}
\end{document}
