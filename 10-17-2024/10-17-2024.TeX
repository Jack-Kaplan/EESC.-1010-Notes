\documentclass[12pt]{article}
\usepackage{amsmath}
\usepackage{graphicx}
\usepackage{hyperref}

\title{Earthquakes: Causes, Distribution, and Hazards}
\author{}
\date{October 17, 2024}

\begin{document}
\maketitle

\section*{Fault Movements}
\begin{itemize}
    \item \textbf{Definition:} Faults are fractures in the Earth's crust where rocks on either side move. This movement is typically vertical, though horizontal shifts also occur.
    \item Vibrations, known as seismic waves, spread outward from fault movements in all directions.
    \item \textbf{Sources of Pressure:}
    \begin{itemize}
        \item Molten rock beneath the Earth's crust exerts pressure, causing instability.
        \item Rising molten rock, often associated with volcanic activity, triggers volcanic earthquakes.
        \item The interaction of liquid and solid materials, such as magma against rock, generates additional vibrations.
    \end{itemize}
    \item \textbf{Focus:} The focus is the underground point where an earthquake originates.
    \item \textbf{Epicenter:} This is the point on the Earth's surface directly above the focus. It is often the location of maximum damage.
    \item The focus and its depth can be calculated using seismic data, aiding in understanding the earthquake's dynamics.
\end{itemize}

\section*{Distribution of Earthquakes}
\begin{itemize}
    \item \textbf{Geographic Zones:} Earthquakes occur most frequently in specific zones:
    \begin{itemize}
        \item The Pacific "Ring of Fire," a horseshoe-shaped region marked by active volcanoes and frequent seismic activity.
        \item A seismic belt extending from Spain through the Mediterranean and into Asia, including regions like Turkey and China.
    \end{itemize}
    \item \textbf{Ocean Ridges:}
    \begin{itemize}
        \item These are underwater mountain ranges where tectonic plates diverge.
        \item Earthquakes along these ridges occur on vertical faults as the plates pull apart.
    \end{itemize}
    \item \textbf{Oceanic Trenches:}
    \begin{itemize}
        \item Found along the edges of continents, these deep depressions are caused by one tectonic plate being forced beneath another.
        \item Earthquake foci near trenches are often distributed along tilted planes, reflecting the angle of subduction.
    \end{itemize}
\end{itemize}

\section*{Earthquake Hazards}
Earthquakes pose significant risks, often magnified by their secondary effects. These include:

\subsection*{1. Ground Shaking}
\begin{itemize}
    \item Ground shaking damages buildings, infrastructure, and human lives.
    \item In Manhattan, skyscrapers are designed with flexibility to withstand wind stress, a principle that also helps resist seismic forces.
    \item \textbf{Base Isolation:} Structures without proper anchoring, such as older brick or brownstone buildings, are particularly vulnerable.
    \item \textbf{Case Studies:}
    \begin{itemize}
        \item Armenia (1988): A magnitude 6.8 earthquake caused widespread destruction, largely due to poor building standards.
        \item San Francisco (1989): Despite a magnitude of 6.9, stricter regulations and better construction mitigated the damage.
    \end{itemize}
    \item \textbf{Mitigation:}
    \begin{itemize}
        \item Zoning laws should prevent construction in high-risk areas.
        \item High-quality materials and engineering practices are essential; poor materials can lead to collapse even with sound designs.
    \end{itemize}
\end{itemize}

\subsection*{2. Fault Movement}
\begin{itemize}
    \item Direct fault movement can destroy even the most robustly built structures.
    \item Zoning laws often prohibit building near active faults, but enforcement and exceptions undermine their effectiveness.
\end{itemize}

\subsection*{3. Ground Failure}
\begin{itemize}
    \item Earthquakes destabilize slopes, leading to landslides or earthslides that destroy infrastructure and disrupt communities.
\end{itemize}

\subsection*{4. Fire}
\begin{itemize}
    \item Fractured natural gas pipelines are a primary cause of post-earthquake fires.
    \item Damaged water supplies can impede firefighting efforts, compounding the disaster.
    \item \textbf{Solution:} Flexible piping systems reduce the risk of breakage during seismic activity.
\end{itemize}

\subsection*{5. Tsunamis}
\begin{itemize}
    \item \textbf{Definition:} Tsunamis are large waves caused by underwater earthquakes, particularly those involving horizontal fault movements.
    \item \textbf{Mitigation:}
    \begin{itemize}
        \item Evacuation plans should direct people to higher ground or reinforced tall buildings.
        \item Detection systems, using strategically placed buoys, monitor wave activity and issue early warnings to prevent loss of life.
    \end{itemize}
\end{itemize}

\end{document}
