\documentclass{article}
\usepackage{amsmath}
\usepackage{graphicx}
\usepackage{array}
\usepackage{tikz}
\usetikzlibrary{shapes, arrows, positioning, matrix}

\title{Earth's Structure and Dynamics}
\date{November 7, 2024}

\begin{document}
\maketitle

\section*{Meteorites and Earth's Composition}

It is not possible to directly observe Earth's deepest layers; drilling and excavation stop far above the mantle. Instead, geoscientists rely on indirect evidence. Meteorites offer one such source of insight. When planets in space are fragmented, the scattered debris often includes materials analogous to the crust, mantle, and core of terrestrial planets. These fragments arrive on Earth as meteorites. Because they formed under conditions similar to those that shaped Earth, their mineral and chemical properties serve as representative samples of what likely exists within our own planet's interior.

Earth materials reach the surface through a few indirect routes. Portions of the mantle may be carried upward by volcanic activity as xenoliths, and large-scale tectonic processes can uplift segments of oceanic crust and attached mantle material, known as ophiolites. By examining such samples, along with meteorites of similar composition, researchers identify minerals like olivine and pyroxene that match seismic data from the mantle. Combining these approaches provides a consistent picture of Earth's layered internal structure.

\section*{Seismic Methods and Internal Boundaries}

Earthquakes generate seismic waves that travel through the planet. These waves change velocity and direction at boundaries between different internal layers. By recording and analyzing how seismic waves reflect and refract, geophysicists determine the presence of distinct zones within Earth. For example, the boundary between the crust and mantle, known as the Mohorovičić Discontinuity (Moho), is revealed by reflections of seismic waves. Differences in wave behavior confirm the layered structure and also indicate which layers are solid, partially molten, or fully liquid.

\section*{Chemical and Physical Layering of Earth}

Earth is typically described in terms of two complementary frameworks: chemical composition and physical properties.

Chemically, Earth is composed of a thin, rocky crust rich in silicates; a denser, silicate mantle containing more iron and magnesium; and an iron-dominated core. The core is split into a molten outer region and a solid inner sphere of iron and nickel.

Physically, Earth can be divided into layers based on their mechanical behavior. The outermost lithosphere includes both the crust and the uppermost, rigid mantle. Beneath this is the asthenosphere, a zone of the upper mantle that is partially molten and capable of slow, ductile flow. Below the asthenosphere lies the mesosphere (lower mantle), which remains solid due to immense pressure. Deeper still, the outer core is liquid iron generating Earth's magnetic field, and the inner core is a solid, dense sphere of iron and nickel.

\begin{table}[h!]
\centering
\begin{tabular}{|l|l|l|}
\hline
\textbf{Layer} & \textbf{Physical State} & \textbf{Composition} \\
\hline
Lithosphere & Rigid solid & Crust + upper mantle \\
Asthenosphere & Plastic/ductile & Upper mantle \\
Mesosphere (Lower Mantle) & Solid & Silicate mantle \\
Outer Core & Liquid & Molten iron (with some nickel) \\
Inner Core & Solid & Iron and nickel \\
\hline
\end{tabular}
\caption{Summary of Earth's Internal Layers}
\end{table}

Seismic waves clarify these properties. P-waves (primary waves) travel through solids, liquids, and gases, while S-waves (secondary waves) move only through solids. This distinction helps identify which layers are molten and which remain solid. For example, the inability of S-waves to pass through the outer core demonstrates that it is liquid.

\section*{Faults and Tectonic Stress}

Faults form when rocks experience stresses that cause them to fracture and move relative to each other. Analyzing faults provides valuable clues about the forces acting on a region of Earth's crust.

Normal faults occur when the crust is stretched. The hanging wall block (the portion of rock that would “hang” overhead if one stood along the fault plane) moves downward relative to the footwall block. This geometry indicates tensile (pull-apart) forces. The East African Rift Valley, a large-scale feature spanning thousands of miles, exemplifies normal faulting as the African continent is pulled apart.

In contrast, reverse faults form when the crust is compressed. In these cases, the hanging wall moves upward relative to the footwall. Such faults are associated with mountain building and intense compressional stresses. The Appalachian Mountains, for instance, contain abundant reverse faults, showing that the region was once subjected to substantial horizontal compression.

A third type of fault, the strike-slip fault, involves horizontal motion rather than vertical displacement. The San Andreas Fault in California is a prominent example, with blocks of crust sliding past each other laterally.
\end{document}
